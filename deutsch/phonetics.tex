\documentclass[12pt]{article}
\usepackage{preamble}

\pagestyle{fancy}
\fancyhead[LO,LE]{Немецкий язык}
\fancyhead[RO,RE]{Deutsche Sprache}

\begin{document}
    \section{I. Фонетика \hfill Phonetik}

    \subsection{I.1 Алфавит \hfill Alphabet}

    В немецком языке используются 26 латинские букв, помимо этого еще 3 гласных с умлаутом (две точки над буквой, \deutscht{ä, ö, ü}) и эсцетт (\deutscht{ß}) - всего 30 букв

    \begin{tabular}{p{0.08\linewidth}|p{0.13\linewidth}|p{0.12\linewidth}|p{0.12\linewidth}|p{0.22\linewidth}|p{0.2\linewidth}}
        Буква             & Название                                                    & Диктовка  & Звуки                                                                         & Примеры слов                                                                     & Примечание            \\
        \hline
        \deutscht{A \, a} & \deutscht{A} \textipa{[a\textlengthmark]}                   & Anton     & \textipa{[a]}, \textipa{[a\textlengthmark]}                       & \deutscht{\textbf{A}pfel}, \deutscht{T\textbf{a}g}  & \\
        \deutscht{Ä \, ä} & \deutscht{A umlaut} \textipa{[\textepsilon\textlengthmark]} & Ärger     & \textipa{[\textepsilon]}, \textipa{[\textepsilon\textlengthmark]}  & \deutscht{H\textbf{ä}nde}, \deutscht{\textbf{Ä}hre} & похоже на \enquote{э} \\
        \deutscht{B \, b} & \deutscht{Be} \textipa{[be\textlengthmark]}                 & Berta     & \textipa{[b]}                                                     & \deutscht{\textbf{B}ruder} & \\
        \deutscht{C \, c} & \deutscht{Ce} \textipa{[tse\textlengthmark]}                & Cäser     & \textipa{[k]}, \textipa{[ts]}                                     & \deutscht{\textbf{C}reme}, \deutscht{\textbf{C}embalo}  & \\
        \deutscht{D \, d} & \deutscht{De} \textipa{[de\textlengthmark]}                 & Dora      & \textipa{[d]}                                                     & \deutscht{\textbf{d}enken} & \\
        \deutscht{E \, e} & \deutscht{E} \textipa{[e\textlengthmark]}                   & Emil      & \textipa{[\textepsilon]}, \textipa{[\textreve]}, \textipa{[e\textlengthmark]}                                          & \deutscht{k\textbf{e}nnen}, \deutscht{b\textbf{e}kannt}, \deutscht{S\textbf{ee}} & \\
        \deutscht{F \, f} & \deutscht{Ef} \textipa{[\textepsilon f]}                    & Friedrich & \textipa{[f]}                                                     & \deutscht{\textbf{F}racht} & \\
        \deutscht{G \, g} & \deutscht{Ge} \textipa{[ge\textlengthmark]}                 & Gustav    & \textipa{[g]}, \textipa{[\textyogh]}, \textipa{[\c{c}]}                                                     & \deutscht{\textbf{g}ut}, \deutscht{Gara\textbf{g}e}, \deutscht{Köni\textbf{g}} & \\
        \deutscht{H \, h} & \deutscht{Ha} \textipa{[ha\textlengthmark]}                 & Heinrich  & \textipa{[h]}                                                  & \deutscht{\textbf{h}eute} & \\
        \deutscht{I \, I} & \deutscht{I} \textipa{[i\textlengthmark]}                   & Ida       & \textipa{[\textsci]}, \textipa{[i\textlengthmark]}                                          & \deutscht{b\textbf{i}tten}, \deutscht{s\textbf{ie}ben} & \\
        \deutscht{J \, j} & \deutscht{Jot} \textipa{[j\textopeno t]}                    & Jakob     & \textipa{[j]}, \textipa{[\textyogh]}                                                 & \deutscht{\textbf{j}ung}, \deutscht{\textbf{J}ournalist} & \\
        \deutscht{K \, k} & \deutscht{Ka} \textipa{[ka\textlengthmark]}                 & Konrad    & \textipa{[k]}                                                  & \deutscht{\textbf{K}ette} & \\
        \deutscht{L \, l} & \deutscht{El} \textipa{[\textepsilon l]}                    & Ludwig    & \textipa{[l]}                                                                 & \deutscht{F\textbf{l}ug} & \\
        \deutscht{M \, m} & \deutscht{Em} \textipa{[\textepsilon m]}                    & Martha    & \textipa{[m]}                                                                 & \deutscht{Sti\textbf{mm}e} & \\
        \deutscht{N \, n} & \deutscht{En} \textipa{[\textepsilon n]}                    & Nordpal   & \textipa{[n]}                                                  & \deutscht{se\textbf{n}den} & \\
        \deutscht{O \, o} & \deutscht{O} \textipa{[o\textlengthmark]}                   & Otto      & \textipa{[\textopeno]}, \textipa{[o\textlengthmark]},                                                                  & \deutscht{\textbf{o}ffen}, \deutscht{K\textbf{o}hl} & \\
        \deutscht{Ö \, ö} & \deutscht{O umlaut} \textipa{[\o\textlengthmark]}           & Ökonom    & \textipa{[\oe]}, \textipa{[\o\textlengthmark]},                                             & \deutscht{\textbf{Ö}sterreich}, \deutscht{zw\textbf{ö}lf} & похоже на \enquote{ё} \\

    \end{tabular}


\end{document}