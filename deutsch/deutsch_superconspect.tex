\documentclass[12pt]{article}
\usepackage{preamble}

\pagestyle{fancy}
\fancyhead[LO,LE]{Немецкий язык}
\fancyhead[RO,RE]{Deutsche Sprache}

\renewcommand{\thesection}{}

\begin{document}

    \tableofcontents
    \clearpage

        % begin deutsch_01_1_phonetics.tex

    \section{I. Фонетика \hfill Phonetik}

    \subsection{I.1 Алфавит \hfill Alphabet}

    В немецком языке используются 26 латинские букв, помимо этого еще 3 гласных с умлаутом (две точки над буквой, \deutscht{ä, ö, ü}) и эсцетт (\deutscht{ß}) - всего 30 букв

    \begin{longtable}{p{0.12\linewidth}|p{0.17\linewidth}|p{0.16\linewidth}|p{0.16\linewidth}|p{0.26\linewidth}}
        Буква             & Название                                                                      & Диктовка  & Звуки                                                                         & Примеры слов                                                                     \\
        \hline
        \deutscht{A \, a} & \deutscht{A} \textipa{[a\textlengthmark]}                                     & Anton     & \textipa{[a]}, \textipa{[a\textlengthmark]}                       & \deutscht{\textbf{A}pfel}, \deutscht{T\textbf{a}g}  \\
        \deutscht{Ä \, ä} & \deutscht{A umlaut} \textipa{[\textepsilon\textlengthmark]}                   & Ärger     & \textipa{[\textepsilon]}, \textipa{[\textepsilon\textlengthmark]}  & \deutscht{H\textbf{ä}nde}, \deutscht{\textbf{Ä}hre} \\
        \deutscht{B \, b} & \deutscht{Be} \textipa{[be\textlengthmark]}                                   & Berta     & \textipa{[b]}                                                     & \deutscht{\textbf{B}ruder} \\
        \deutscht{C \, c} & \deutscht{Ce} \textipa{[tse\textlengthmark]}                                  & Cäser     & \textipa{[k]}, \textipa{[ts]}                                     & \deutscht{\textbf{C}reme}, \deutscht{\textbf{C}embalo}  \\
        \deutscht{D \, d} & \deutscht{De} \textipa{[de\textlengthmark]}                                   & Dora      & \textipa{[d]}                                                     & \deutscht{\textbf{d}enken} \\
        \deutscht{E \, e} & \deutscht{E} \textipa{[e\textlengthmark]}                                     & Emil      & \textipa{[\textepsilon]}, \textipa{[\textreve]}, \textipa{[e\textlengthmark]}                                          & \deutscht{k\textbf{e}nnen}, \deutscht{b\textbf{e}kannt}, \deutscht{S\textbf{ee}} \\
        \deutscht{F \, f} & \deutscht{Ef} \textipa{[\textepsilon f]}                                      & Friedrich & \textipa{[f]}                                                     & \deutscht{\textbf{F}racht} \\
        \deutscht{G \, g} & \deutscht{Ge} \textipa{[ge\textlengthmark]}                                   & Gustav    & \textipa{[g]}, \textipa{[\textyogh]}, \textipa{[\c{c}]}                                                     & \deutscht{\textbf{g}ut}, \deutscht{Gara\textbf{g}e}, \deutscht{Köni\textbf{g}} \\
        \deutscht{H \, h} & \deutscht{Ha} \textipa{[ha\textlengthmark]}                                   & Heinrich  & \textipa{[h]}                                                  & \deutscht{\textbf{h}eute} \\
        \deutscht{I \, I} & \deutscht{I} \textipa{[i\textlengthmark]}                                     & Ida       & \textipa{[\textsci]}, \textipa{[i\textlengthmark]}                                          & \deutscht{b\textbf{i}tten}, \deutscht{s\textbf{ie}ben} \\
        \deutscht{J \, j} & \deutscht{Jot} \textipa{[j\textopeno t]}                                      & Jakob     & \textipa{[j]}, \textipa{[\textyogh]}                                                 & \deutscht{\textbf{j}ung}, \deutscht{\textbf{J}ournalist} \\
        \deutscht{K \, k} & \deutscht{Ka} \textipa{[ka\textlengthmark]}                                   & Konrad    & \textipa{[k]}                                                  & \deutscht{\textbf{K}ette} \\
        \deutscht{L \, l} & \deutscht{El} \textipa{[\textepsilon l]}                                      & Ludwig    & \textipa{[l]}                                                                 & \deutscht{F\textbf{l}ug}                                                         \\
        \deutscht{M \, m} & \deutscht{Em} \textipa{[\textepsilon m]}                                      & Martha    & \textipa{[m]}                                                                 & \deutscht{Sti\textbf{mm}e}                                                       \\
        \deutscht{N \, n} & \deutscht{En} \textipa{[\textepsilon n]}                                      & Nordpal   & \textipa{[n]}                                                  & \deutscht{se\textbf{n}den} \\
        \deutscht{O \, o} & \deutscht{O} \textipa{[o\textlengthmark]}                                     & Otto      & \textipa{[\textopeno]}, \textipa{[o\textlengthmark]},                                                                  & \deutscht{\textbf{o}ffen}, \deutscht{K\textbf{o}hl} \\
        \deutscht{Ö \, ö} & \deutscht{O umlaut} \textipa{[\o\textlengthmark]}                             & Ökonom    & \textipa{[\oe]}, \textipa{[\o\textlengthmark]},                                             & \deutscht{\textbf{Ö}sterreich}, \deutscht{zw\textbf{ö}lf} \\
        \deutscht{P \, p} & \deutscht{Pe} \textipa{[pe\textlengthmark]}                                   & Paula     & \textipa{[p]}                                                  & \deutscht{\textbf{P}unkt}  \\
        \deutscht{Q \, q} & \deutscht{Qu} \textipa{[ku\textlengthmark]}                                   & Quelle    & \textipa{[kv]}                                                                             & \deutscht{\textbf{Q}uadrat} \\
        \deutscht{R \, r} & \deutscht{Er} \textipa{[\textepsilon r]}                                      & Richard   & \textipa{[r]}                                                  & \deutscht{D\textbf{r}ache} \\
        \deutscht{S \, s} & \deutscht{Es} \textipa{[\textepsilon s]}                                      & Samuel    & \textipa{[s]}, \textipa{[z]}                                                  & \deutscht{Bu\textbf{s}}, \deutscht{\textbf{s}ehen} \\
        \deutscht{ẞ \, ß} & \deutscht{Eszett} \textipa{[\textepsilon s"\texttoptiebar{ts}\textepsilon t]} & Eszett    & \textipa{[s]}                                                  & \deutscht{hei\textbf{ß}} \\
        \deutscht{T \, t} & \deutscht{Te} \textipa{[te\textlengthmark]}                                   & Theodor   & \textipa{[t]}                                                  & \deutscht{Pla\textbf{tt}e} \\
        \deutscht{U \, u} & \deutscht{U} \textipa{[u\textlengthmark]}                                     & Ulrich    & \textipa{[\textupsilon]}, \textipa{[u\textlengthmark]}                                                  & \deutscht{\textbf{u}nter}, \deutscht{\textbf{U}hr} \\
        \deutscht{Ü \, ü} & \deutscht{U umlaut} \textipa{[y\textlengthmark]}                              & Übermut   & \textipa{[\textscy]}, \textipa{[y\textlengthmark]}                                                  & \deutscht{\textbf{Ü}bung}, \deutscht{k\textbf{ü}ssen} \\
        \deutscht{V \, v} & \deutscht{Vau} \textipa{[fa\textupsilon]}                                     & Viktor    & \textipa{[f]}, \textipa{[v]}                                                  & \deutscht{\textbf{V}ater}, \deutscht{\textbf{V}ase} \\
        \deutscht{W \, w} & \deutscht{We} \textipa{[ve\textlengthmark]}                                   & Wilheim   & \textipa{[v]}                                                  & \deutscht{\textbf{W}olken} \\
        \deutscht{X \, x} & \deutscht{Ix} \textipa{[iks\textupsilon]}                                     & Xanthippe & \textipa{[ks]}                                                  & \deutscht{\textbf{X}ylophon} \\
        \deutscht{Y \, y} & \deutscht{Ypsilon} \textipa{["\textscy psil\textopeno n]}                     & Ypsilon   & \textipa{[\textscy]}, \textipa{[y\textlengthmark]}, \textipa{[j]}                                                  & \deutscht{\textbf{Y}psilon}, \deutscht{T\textbf{y}p}, \deutscht{\textbf{Y}acht} \\
        \deutscht{Z \, z} & \deutscht{Zet} \textipa{[\texttoptiebar{ts}\textepsilon t]}                   & Zacharias & \textipa{[\texttoptiebar{ts}]}                                                  & \deutscht{drei\textbf{z}ehn} \\

    \end{longtable}
    % end deutsch_01_1_phonetics.tex

    % begin deutsch_02_1_noun.tex

    \section{II. Части речи \hfill Wortarten}

    \subsection{II.1 Имя существительное \hfill Substantiv}
    % end deutsch_02_1_noun.tex

    % begin deutsch_02_2_adjective.tex

    \subsection{II.2 Имя прилагательное \hfill Adjektiv}
    % end deutsch_02_2_adjective.tex



\end{document}

